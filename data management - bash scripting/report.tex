\documentclass{article}
\usepackage[utf8]{inputenc}

\title{\textbf{CountReviews Shell Script}}
\author{Yaqin Kayem Hasan \\ ID: ykh1e19}
\date{March 2021}

\begin{document}

\maketitle

\clearpage

The following is the shell script in the final submission with comments omitted and line labels added for future reference: 
\begin{verbatim}
line 1: #!/bin/bash
line 2: cd $1
line 3: for file in *; do
line 4:   filename=$(basename $file .dat)
line 5:   n=$(grep -c "<Author>" $file)
line 6:   v="$v$filename $n\n"
line 7: done
line 8: echo -n -e $v | sort -rnk 2
\end{verbatim}
\\
Line 1 is included in every bash script, to specify that it is a bash script. Line 2 navigates to the directory passed in as the first argument. Line 3 through line 7 is a for-loop that goes through every file in the specified directory and performs the following actions: \\

First, in line 4, it removes the .dat extension from the file name and stores that as a variable. Second, in line 5, it uses grep -c to count the number of instances where "\textless Author\textgreater " appears in the file, which corresponds with the number of reviews. Third, it appends both the filename without the extension and the number of reviews to an arbitrary variable. \\

After the for-loop, line 8 sorts the contents of the arbitrary variable by the second key (the number of reviews) and prints it out. 

\end{document}
